\documentclass[11pt]{beamer}
\usetheme{Copenhagen}
\usepackage[utf8]{inputenc}
\usepackage[english]{babel}
\usepackage{amsmath}
\usepackage{amsfonts}
\usepackage{amssymb}
\usepackage{graphicx}
\DeclareMathOperator{\argmin}{\mathrm{arg\, min}}
\author{Onofre Martorell, Lidia Talavera}
\title{Image inpainting}
%\setbeamercovered{transparent} 
%\setbeamertemplate{navigation symbols}{} 
%\logo{} 
%\institute{} 
%\date{} 
%\subject{} 
\begin{document}

\begin{frame}
\titlepage
\end{frame}

%\begin{frame}
%\tableofcontents
%\end{frame}

\begin{frame}{From functional to Laplace equation}
We have the functional
$$
\displaystyle\argmin\limits_{u\in W^{1,2}(\Omega)}\int _D |\nabla u(x)|^2dx
$$
According to the Necessary condition for the extremum\footnote{We saw this result at the first class of the module}, the minimum of the functional is the solution of
$$-\sum_{i=1}^2 \frac{\partial}{\partial x_i}\frac{\partial\mathcal{F}}{\partial p_i} + \frac{\partial\mathcal{F}}{\partial u} = 0$$

where $\nabla u(u) = (p_1, p_2)$ and $\mathcal{F}$ is the functional.
\end{frame}

\begin{frame}{From functional to Laplace equation}
Replacing the given functional, we get
$$ 0 = - \frac{\partial}{\partial x}\frac{\partial(u_x^2 + u_y^2)}{\partial u_x} - \frac{\partial}{\partial y}\frac{\partial(u_x^2 + u_y^2)}{\partial u_y} + \frac{\partial(u_x^2 + u_y^2)}{\partial u} = $$
$$ = - \frac{\partial}{\partial x}(2u_x) - \frac{\partial}{\partial y}(2u_y) + 0 = 2u_{xx} + 2u_{yy}$$
In conclusion,
$$2(u_{xx} + u_{yy})=0\Longrightarrow \Delta u = 0$$
\end{frame}

\begin{frame}{The problem of inpainting}
The problem of inpainting can be modeled as
$$
\begin{cases}
\displaystyle\argmin\limits_{u\in W^{1,2}(\Omega)}\int _D |\nabla u(x)|^2dx,\\
u|_{\partial D} = f
\end{cases}
$$
where $f$ is the image to inpaint.


With the result obtained, this problem is equivalent to find the solution of
$$\left\lbrace
\begin{array}{l l}

\Delta u = 0& \text{in }D \\
u = f & \text{in }\partial D

\end{array}\right.
$$
The equation is completed with homogeneous Neumann boundary
conditions at the boundary of the image.

\end{frame}

\begin{frame}
The laplacian can be computed as
$$\Delta u = div(\nabla u).$$

\begin{block}{Discretization of the gradient}
Using forward differences, we obtain
$$\nabla u_{ij} = \bigg(\frac{u_{i+1,j} - u_{i,j}}{h_i}, \frac{u_{i,j+1} - u_{i,j}}{h_j}\bigg)$$
\end{block}
\begin{block}{Discretization of the divergence}
Taking $v = (v_1, v_2)$ and using backward differences, we obtain
$$div(v)_{i,j} = \frac{v_{1_{i,j}} - v_{1_{i-1,j}}}{h_i} + \frac{v_{2_{i,j}} - v_{2_{i,j-1}}}{h_j}$$
\end{block}
\end{frame}

\begin{frame}
Joining the previous discretizations and doing some calculus we obtain the following equation for each pixel in $D$
$$\frac{1}{h_j^2}u_{i,j-1} + \frac{1}{h_i^2}u_{i-1, j} - \bigg(\frac{2}{h_i^2}+\frac{2}{h_j^2}\bigg)u_{i,j} + \frac{1}{h_i^2}u_{i+1,j} + \frac{1}{h_j^2}u_{i, j+1} = 0$$


\end{frame}

\begin{frame}
Boundary conditions
\end{frame}

\begin{frame}
Equations for all the points of the image
\end{frame}

\begin{frame}
Results(images)
\end{frame}
\end{document}